%%
%% Hyperbolic Trig Identities
%%
%% Author: William Gao
%% 09/04/2019
%%

\documentclass[11pt]{article}
\setlength{\parindent}{0pt}                  % disable indent globally

\RequirePackage[
    letterpaper,
    left=0.8in,
    right=0.8in,
    top=0.70in,
    bottom=0.55in,
    nohead
]{geometry}

% Packages
\usepackage{amssymb}
\usepackage[fleqn]{amsmath}
\usepackage{hyperref}
\hypersetup{
    colorlinks=true,
    urlcolor=blue
}

\newcommand{\hzline}{\noindent\makebox[\linewidth]{\rule{0.8\paperwidth}{0.4pt}}}

\title{Hyperbolic Trig Identities}
%\author{math2.org}
\author{Version 1}
\date{4 September 2019}

\begin{document}

\maketitle

\section{Introduction}
A compiled list of the most commonly used hyperbolic trig identities. Enjoy. \\

All credits goes to \href{http://math2.org/math/trig/hyperbolics.htm}{math2.org}. If you have comments, corrections, or clarifications, please submit an issue or pull request on \href{https://github.com/chemclub/calculus}{GitHub}.


\section{Hyperbolic Definitions}

\begin{align*}
& \sinh x = \frac{e^x-e^{-x}}{2} \\ 
& \operatorname{csch}x = \frac{1}{\sinh x} = \frac{2}{e^x-e^{-x}}
\end{align*}

\begin{align*}
& \sinh x = \frac{e^x-e^{-x}}{2} \\
& \operatorname{csch}x = \frac{1}{\sinh x} = \frac{2}{e^x-e^{-x}}
\end{align*}

\begin{align*}
& \cosh x = \frac{e^x+e^{-x}}{2} \\
& \operatorname{sech}x = \frac{1}{\cosh x} = \frac{2}{e^x+e^{-x}}
\end{align*}

\begin{align*}
& \tanh x = \frac{\sinh x}{\cosh x} = \frac{e^x-e^{-x}}{e^x-e^{-x}} \\
& \operatorname{coth}x = \frac{1}{\tanh x} = \frac{e^x+e^{-x}}{e^x-e^{-x}}
\end{align*}


\begin{align*}
& \cosh^2x-\sinh^2x = 1 \\
& \tanh^2x+\operatorname{sech}^2x \\
& \coth^2x-\operatorname{csch}^2x
\end{align*}


\section{Inverse Hyperbolic Definitions}

\begin{align*}
& \operatorname{arcsinh}z=\ln\left(z+\sqrt{z^2+1}\right) \\
& \operatorname{arccosh}z=\ln\left(z\pm\sqrt{z^2-1}\right) \\
& \operatorname{arctanh}z=\frac{1}{2}\ln\left(\frac{1+z}{1-z}\right)
\end{align*}

\begin{align*}
& \operatorname{arccsch}z=\ln\left(\frac{1+\sqrt{1+z^2}}{z}\right) \\
& \operatorname{arcsech}z=\ln\left(\frac{1+\sqrt{1-z^2}}{z}\right) \\
& \operatorname{arccoth}z=\frac{1}{2}\ln\left(\frac{z+1}{z-1}\right)
\end{align*}


\section{Relations to Trigonometric Functions}

\begin{align*}
& \sinh z=-i\sin\left(iz\right) \\
& \operatorname{csch}z=i\csc\left(iz\right)
\end{align*}

\begin{align*}
& \cos z=\cos\left(iz\right) \\
& \operatorname{sech}z=\sec\left(iz\right)
\end{align*}

\begin{align*}
& \tanh z=-i\tan\left(iz\right) \\
& \coth z=i\cot\left(iz\right)
\end{align*}

\end{document}

